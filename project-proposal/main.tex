\documentclass[12pt]{article}

\usepackage{fullpage}
\usepackage{datetime}

\usepackage{biblatex}
\addbibresource{./references.bib}

\title{
  Learning Neural Orientation Field for Volumetric Hair Reconstruction \\
  {
    \small
    Computer Vision Project Proposal
  }
}
\author{
  Fangjun Zhou \\ fzhou48
  \and Weiran Xu \\ weiran
  \and Zhenyu Zhang \\ zhenyuz5
}
\date{\today}

\begin{document}
  \maketitle

  \section{Introduction}
  % Project motivation

  Reconstructing human hair is one of the most challenging yet critical process in rendering photorealistic digital human. Unlike other parts of the human body, human hair is highly detailed and often intertwined together. Therefore, it's difficult to use traditional photogrammetry method to reconstruct its structure.

  Before machine learning is used in this field, artists often hand crafted splines on skulls to represent hair strands. Each strand is then textured and rendered to mimic the hair volume. This workflow requires a lot of experience as it's non-trivial for artists to infer the final render result from hair stand splines. To reduce the workload and improve the accuracy of hair reconstruction, machine learning models are used to generate hair strand from captured photos.

  \section{Related Work}

  Previous attempt to achieve this goal mainly focus on learning based hair strand generation. This includes some studies about single view hair synthesis \cite{saito_3d_2018, zheng_hairstep_2023, wu_neuralhdhair_2022, ma_single-view_nodate}.

  \section{Method}

  \section{Experiment}
  % Intended experiment

  \printbibliography

\end{document}
