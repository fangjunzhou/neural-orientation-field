\documentclass[12pt]{article}

\usepackage{fullpage}
\usepackage{datetime}

\usepackage{biblatex}
\addbibresource{./references.bib}

\title{
  Learning Neural Orientation Field for Volumetric Hair Reconstruction \\
  {
    \small
    Computer Vision Project Proposal
  }
}
\author{
  Fangjun Zhou \\ fzhou48
  \and Weiran Xu \\ weiran
  \and Zhenyu Zhang \\ zhenyuz5
}
\date{\today}

\begin{document}
  \maketitle

  \section{Introduction}
  % Project motivation

  Reconstructing human hair is one of the most challenging yet critical process in rendering photorealistic digital human. Unlike other parts of the human body, human hair is highly detailed and often intertwined together. Therefore, it's difficult to use traditional photogrammetry method to reconstruct its structure.

  Before machine learning is used in this field, artists often hand crafted splines on skulls to represent hair strands. Each strand is then textured and rendered to mimic the hair volume. This workflow requires a lot of experience as it's non-trivial for artists to infer the final render result from hair stand splines. To reduce the workload and improve the accuracy of hair reconstruction, machine learning models are used to generate hair strand from captured photos.

  \section{Related Work}

  Previous attempt to achieve this goal mainly focus on learning based hair strand generation. This includes some studies about single view hair synthesis \cite{saito_3d_2018, zheng_hairstep_2023, wu_neuralhdhair_2022, ma_single-view_nodate}. Since the image only contains hair structure from one viewing angle, it's impossible to reconstruct entire hair accurately. These models often use pretrained image encoders such as ResNet-50 \cite{saito_3d_2018} to encode the abstract hair style into a feature vector, then use generative models such as U-Net \cite{zheng_hairstep_2023}, VAE \cite{saito_3d_2018}, and diffusion \cite{sklyarova_neural_2023}. These models also struggle with generating curly hair as there's only limited information about growing direction after feature extraction.

  In \cite{sklyarova_neural_2023} and \cite{rosu_neural_2022}, the authors also tried hair syntheses from multi-view images. However, these two studies still failed to capture finer detail.

  Another study about this topic tried to tackle this problem by expanding the traditional PatchMatch MVS (PMVS) algorithm to a Line-based PatchMatch MVS (LPMVS) \cite{nam_strand-accurate_nodate}. This method, despite its high accuracy, doesn't capture the volumetric property of human hair.

  \section{Method}

  \section{Experiment}
  % Intended experiment
  

  \printbibliography

\end{document}
